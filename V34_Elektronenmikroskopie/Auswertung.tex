% !TeX encoding = UTF-8
% !TeX root = V34_TEM.tex
% !TeX spellcheck = de_DE_frami

Unterfokus: Objekt unter Fokusebene (durch geringere Brennweite = stärkere Anregung der Spulen), im Bild 1. Saum dominierend hell (Fresnelsaum)

Überfokus analog

Diffractogramm: Fouriertrafo der Abbildung (aber wegen überlagerung mit ... schlechteres bild als im echten beugungsmodus)

Dunkler Ring im Diffractogramm normal, bei zweizähligem Astigmatismus wäre es zur Ellipse verzerrt, wird durch 2 Stigmatoren (Multipole) korrigiert


Aufnahmen:
Unter-, Über-, Fokus (F, Ü, U)
High resolution (mehrere bilder mit Netz ebenen)
Beugung (verdeckter Null Strahl zum Schutz der Kamera)
Bright, Dark Field

Diffractogramme nicht auf USB Stick dabei: freies Programm ImageJ, wahrscheinlich auch matlab (2D fft)

Meist nur eine Schar Netzebenen sichtbar (anstatt kariert) da die meisten Körner nicht in der fokusebene orientiert sind und somit nur eine Achse passt


Beugung: gewisse ringe auf unterschiedlichen Bildern an besten sichtbar, überlagern (entweder nur Messwerte oder Bilder)
Bei 340mm Kamera abstand (30000x Vergrößerung)
300mm Schirm Abstand (26500x Vergrößerung)

Brightfield: objektivblende lässt nur Hauptstrahl durch
Darkfield: ... nur Ausschnitt aus erstem beugungsRing durch
Zwei Bilder für unterschiedlichen Azimutwinkel --> unterschiedliche cluster leuchten auf (zwei verschiedene Orientierungen)




2. Probe
- convergent beam (Kreise mit gebogenem Linien, geben Aufschluss über dicke der probe)
- Beugung, gleiche Einstellungen wie bei der ersten probe

110 Orientierung
1-1+-1, -11+-1, ..., 002, ..., 004 Reflexe zu sehen



Bei der ersten probe
Träger = Kupfernetz + Kohlenstoff-Folie
Probe = kristallines Pulver (durch verdunsten des lösungsmittels gleichmäßig verteilt)

Erste probe Edelmetall (zu bestimmen, vermutlich Gold), zweite probe Si Kristall



