% !TeX encoding = UTF-8
% !TeX root = V39_Spectroscopy_Rb.tex
% !TeX spellcheck = en_US

\section{Execution} % Experimental setup ?

\subsection{Frequency modulation spectroscopy}
The laser is modulated with a signal
\begin{equation}
 S_1 = a_1 \sin(\omega t)
 \intertext{with}
 \omega = 2 \pi \nu, \nu = 0.1 \mega\hertz
\end{equation}

The laser frequency follows, shifting by $\Delta \nu$ each period.

The signal at the photodiode thus varies by $\Delta I = \ddp{I}{\nu} \cdt \Delta \nu$, allowing us to obtain the first derivative of the spectrum.


\section{Analysis}
\subsection{Absorption spectroscopy}
% TEK0001
% Scale

\subsection{Saturation spectroscopy}








% 1st & 4th peak correspond to 87 Rb
% 2nd & 3rd peak correspond to 85 Rb
% two peaks each from the hyperfine structure of 5s_{1/2} : F = I \pm 1/2

% 87 Rb : 5p_{3/2} --> 5s_{1/2} F \in \{2, 3\}
% 87 Rb : 5p_{3/2} --> 5s_{1/2} F \in \{1, 2\}


% Aufgabe 1: Skalierung anhand zweier Literaturwerte
% Aufgabe 2: Dopplereffekt: Breite --> Temperatur
% Aufgabe 3: restliche Werte incl. Feinstruktur abgleichen

\section{Discussion of errors}
