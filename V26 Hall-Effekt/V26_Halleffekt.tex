% !TeX encoding = UTF-8
% !TeX root = V26_Halleffekt.tex
% !TeX spellcheck = de_DE_frami

\documentclass[a4paper, parskip=half, 12pt, bibliography=totocnumbered]{scrartcl}

\def \deutsch {}
\def \pgroup {1035}
\def \studenti {Trung-An Mach}
\def \emaili {trung-an.mach@uni-ulm.de}
\def \studentii {Sebastian K\"{o}lle}
\def \emailii {sebastian.koelle@uni-ulm.de}
%
\def \deutsch {}
%
% Change these to the members of your group:
%
%\def \gpgroup {group-number}
%\def \studenti {Person1}
%\def \emaili {Email1@uni-ulm.de}
%\def \studentii {Person2}
%\def \emailii {Email2@uni-ulm.de}
% % \def \studentiii { }	% if defined, the layout is adapted to three persons
% % \def \emailiii { }
%
% select the language of the report:
% \def \francais {}
% \def \english {}
% \def \deutsch {}

% !TeX encoding = UTF-8

%______________________________________
%
% This is a collection of packages and macros which I use for lab reports (Protokolle im Grundpraktikum / compte rendu en TP). In particular, it includes an extended version of my macro Titelseite, which has the following syntax:
%	\Titelseite{Praktikum}{Nr.}{Versuch}{Versuchsdatum}{Erstabgabe}{Betreuer}{Email Betreuer}
% If you want no list of tables, write \Titelseite[] instead of \Titelseite
%
% I use neither biblatex nor hyperref as these might be undesired and would have to be added after all other packages. The pgfplot-package is extremely usefull when used properly but wastes ressources if unused. If you need them, uncomment them.
%
% Usage example:
%	\documentclass[paper=a4, parskip, 12pt, DIV = 14]{scrartcl}
%	% Change these to the members of your group:
%
%\def \gpgroup {group-number}
%\def \studenti {Person1}
%\def \emaili {Email1@uni-ulm.de}
%\def \studentii {Person2}
%\def \emailii {Email2@uni-ulm.de}
% % \def \studentiii { }	% if defined, the layout is adapted to three persons
% % \def \emailiii { }
%
% select the language of the report:
% \def \francais {}
% \def \english {}
% \def \deutsch {}

% !TeX encoding = UTF-8

%______________________________________
%
% This is a collection of packages and macros which I use for lab reports (Protokolle im Grundpraktikum / compte rendu en TP). In particular, it includes an extended version of my macro Titelseite, which has the following syntax:
%	\Titelseite{Praktikum}{Nr.}{Versuch}{Versuchsdatum}{Erstabgabe}{Betreuer}{Email Betreuer}
% If you want no list of tables, write \Titelseite[] instead of \Titelseite
%
% I use neither biblatex nor hyperref as these might be undesired and would have to be added after all other packages. The pgfplot-package is extremely usefull when used properly but wastes ressources if unused. If you need them, uncomment them.
%
% Usage example:
%	\documentclass[paper=a4, parskip, 12pt, DIV = 14]{scrartcl}
%	\input{preamble}
%	\begin{document}
%	\Titelseite{\GP}{6}{G-Modul}{14. Dezember 2012}{Erstabgabe}{Wolfgang Limmer}{wolfgang.limmer@uni-ulm.de}
%	\section{Einleitung}
%
% As you see, you can directly afterwards begin with the content of your report!
%
% \copyright ~ Sebastian Kölle 2013
%______________________________________


\usepackage{lmodern}					% Type1-font
\usepackage[utf8]{inputenc}
\usepackage[T1]{fontenc}
\usepackage{textcomp}
\usepackage{microtype}					% less unnecessary hyphenation
\usepackage[frenchb, ngerman, british]{babel}
\usepackage{scrpage2}					% improved header
\usepackage{dsfont}						% letters with double line, e.g. |N
\usepackage{amsmath, amssymb}			% more math
\usepackage{esint}						% circular and multiple integrals
\usepackage[squaren]{SIunits}				% unit support (siunitx might be better)
\usepackage{booktabs}					% nice tables
\usepackage{enumitem}
\usepackage{url}
\usepackage{graphicx}						% include pictures
\usepackage{subfig}						% pictures side by side as a), b)
%\usepackage{lscape, rotating, multirow}		% oversized tables
%\usepackage[backend=biber, sortlocale=de_DE]{biblatex}

%\usepackage{pgfplots}
%\pgfplotsset{compat=1.7, /pgf/number format/.cd, use comma, 1000 sep={}}

%\usepackage[colorlinks=false]{hyperref}		% has to come last

\newcommand{\dfe}[3]{%
	\ifdefined\deutsch #1\fi \ifdefined\francais #2\fi \ifdefined\english #3\fi}

\pagestyle{scrheadings}
\setheadsepline[\textwidth]{1pt}
\automark{section}
\ihead{\textsc{\headmark}}
\chead{}
\cfoot{}
\ohead{\dfe{Seite}{page}{page} {\pagemark}}

\numberwithin{equation}{section}
\linespread{1.1}
\BeforeStartingTOC[toc]{\linespread{1}}	%dense table of contents
%\lefthyphenmin=3					%prevent splitting off short syllables,
%\righthyphenmin=3					%must be repeated after \end{otherlanguage}

\newcommand{\RM}[1]{\MakeUppercase{\romannumeral #1}}	%roman numerals
\newcommand{\LabView}{LabView\texttrademark}
\newcommand{\bruch}[2]{^{#1}\!\!\!/_{\!#2} \,}
\newcommand{\Div}[1]{\operatorname{div} \vec{#1}}
\newcommand{\rot}[1]{\operatorname{rot} \vec{#1}}
\newcommand{\grad}[1]{\operatorname{grad} #1}
\newcommand{\vect}[1]{\mathop{#1}\limits^{\vbox to -.6ex{\kern-0.75ex\hbox{$\rightharpoonup$}\vss}}}
\renewcommand{\d}{\mathrm{d}}
\newcommand{\dd}[1]{\tfrac{\d}{\d#1}}				%differential by...
\newcommand{\ddp}[2]{\frac{\partial#1}{\partial#2}}		%partial differential
\newcommand{\dds}[2]{\tfrac{\partial#1}{\partial#2}}
\newcommand{\somit}{\qquad\Longrightarrow\qquad}	%implies (Folgepfeil)
\makeatletter
\newcommand{\xRArrow}[2][]{\ext@arrow 0955{\arrowfill@{}\Relbar\Rightarrow}{#1}{#2}}
\makeatother
\newcommand{\cdt}{\!\cdot\!}				% less space around \cdot
\def \dens {\tfrac{\kilo\gram\,}{\metre^3}}		% unit of density: kg/m³

\newcommand{\with} {\qquad\text{\dfe{mit}{avec}{with}}\qquad}
\newcommand{\median}[1]{\ensuremath{\langle {#1} \rangle}}
\newcommand{\bra}[1]{\ensuremath{\langle {#1}|}}
\newcommand{\ket}[1]{\ensuremath{|{#1}\rangle}}
\newcommand{\braket}[2]{\ensuremath{\langle {#1}|{#2}\rangle}}

\def \GP {Grundpraktikum der Physik}
\def \FP {Fortgeschrittenenpraktikum}
\def \TP {TP de physique }				% ajouter la matière
\def \Ces {TP Césire }

% titlepage and table of contents

\newif\iflot
\newcommand{\Titelseite}[8][\lottrue] {
\dfe{\selectlanguage{ngerman}}{\selectlanguage{frenchb}}{\selectlanguage{british}}
\begin{titlepage}
	{\flushright
	\includegraphics[width=\textwidth]{../logo_50.jpg} \\}
	%logo of your university, width might have to be adjusted
	\vspace{5em}
	{\centering
	{\huge \textbf{#2} } \\
	\vspace{4em}
	{\large \textbf{\textsf{\dfe{Versuch}{expérience}{experience} #3:~#4}} } \\
	\vspace{2em}
	{\large \dfe{Durchführung}{exécution}{execution}: #5 \\
	#6: \today \\} }
	
	\vspace{3em}
	{\large
	\begin{tabbing}
		\Large \dfe{Gruppe}{groupe}{group} \pgroup: \\[1mm]
		\studenti \qquad~~ \= \texttt{\emaili}\\
		\studentii  \> \texttt{\emailii}\\
		\ifdefined\studentiii		\studentiii  \> \texttt{\emailiii}\\ \fi
		\\
		\Large \dfe{Betreuer}{enseignant}{supervisor}:\\[1mm]
		#7 \qquad \> \texttt{#8}
	\end{tabbing}}
	\vspace{2em}

	\dfe
	{Wir bestätigen hiermit, dieses Protokoll selbstständig erarbeitet zu haben und um dessen gesamten Inhalt zu wissen. Zur Ausarbeitung wurden ausschließlich die angegebenen Quellen und Hilfen in Anspruch genommen.}
	{Nous attestons d'avoir produit ce compte rendu nous-mêmes. Nous n'avons utilisé que les ressources et sources indiquées.}
	{We hereby attest having written this lab report ourselves. We used no sources aside from those we indicated.}

	\vspace{2em}
	$\overline{\makebox[4.8cm][c]{\raisebox{0pt}[3ex]{\studenti}}}
	\ifdefined\studentiii \hspace{1cm} \else \hfill \fi
	\overline{\makebox[4.8cm][c]{\raisebox{0pt}[3ex]{\studentii}}}
	\ifdefined\studentiii \hspace{1cm}
	\overline{\makebox[4.8cm][c]{\raisebox{0pt}[3ex]{\studentiii}}} \fi$

	\thispagestyle{empty}
	\setcounter{page}{0}
\end{titlepage}


\tableofcontents
\listoffigures
#1 \iflot \listoftables \fi	% lot can be suppressed by empty optional parameter
\enlargethispage{5\baselineskip}
\newpage
}
%	\begin{document}
%	\Titelseite{\GP}{6}{G-Modul}{14. Dezember 2012}{Erstabgabe}{Wolfgang Limmer}{wolfgang.limmer@uni-ulm.de}
%	\section{Einleitung}
%
% As you see, you can directly afterwards begin with the content of your report!
%
% \copyright ~ Sebastian Kölle 2013
%______________________________________


\usepackage{lmodern}					% Type1-font
\usepackage[utf8]{inputenc}
\usepackage[T1]{fontenc}
\usepackage{textcomp}
\usepackage{microtype}					% less unnecessary hyphenation
\usepackage[frenchb, ngerman, british]{babel}
\usepackage{scrpage2}					% improved header
\usepackage{dsfont}						% letters with double line, e.g. |N
\usepackage{amsmath, amssymb}			% more math
\usepackage{esint}						% circular and multiple integrals
\usepackage[squaren]{SIunits}				% unit support (siunitx might be better)
\usepackage{booktabs}					% nice tables
\usepackage{enumitem}
\usepackage{url}
\usepackage{graphicx}						% include pictures
\usepackage{subfig}						% pictures side by side as a), b)
%\usepackage{lscape, rotating, multirow}		% oversized tables
%\usepackage[backend=biber, sortlocale=de_DE]{biblatex}

%\usepackage{pgfplots}
%\pgfplotsset{compat=1.7, /pgf/number format/.cd, use comma, 1000 sep={}}

%\usepackage[colorlinks=false]{hyperref}		% has to come last

\newcommand{\dfe}[3]{%
	\ifdefined\deutsch #1\fi \ifdefined\francais #2\fi \ifdefined\english #3\fi}

\pagestyle{scrheadings}
\setheadsepline[\textwidth]{1pt}
\automark{section}
\ihead{\textsc{\headmark}}
\chead{}
\cfoot{}
\ohead{\dfe{Seite}{page}{page} {\pagemark}}

\numberwithin{equation}{section}
\linespread{1.1}
\BeforeStartingTOC[toc]{\linespread{1}}	%dense table of contents
%\lefthyphenmin=3					%prevent splitting off short syllables,
%\righthyphenmin=3					%must be repeated after \end{otherlanguage}

\newcommand{\RM}[1]{\MakeUppercase{\romannumeral #1}}	%roman numerals
\newcommand{\LabView}{LabView\texttrademark}
\newcommand{\bruch}[2]{^{#1}\!\!\!/_{\!#2} \,}
\newcommand{\Div}[1]{\operatorname{div} \vec{#1}}
\newcommand{\rot}[1]{\operatorname{rot} \vec{#1}}
\newcommand{\grad}[1]{\operatorname{grad} #1}
\newcommand{\vect}[1]{\mathop{#1}\limits^{\vbox to -.6ex{\kern-0.75ex\hbox{$\rightharpoonup$}\vss}}}
\renewcommand{\d}{\mathrm{d}}
\newcommand{\dd}[1]{\tfrac{\d}{\d#1}}				%differential by...
\newcommand{\ddp}[2]{\frac{\partial#1}{\partial#2}}		%partial differential
\newcommand{\dds}[2]{\tfrac{\partial#1}{\partial#2}}
\newcommand{\somit}{\qquad\Longrightarrow\qquad}	%implies (Folgepfeil)
\makeatletter
\newcommand{\xRArrow}[2][]{\ext@arrow 0955{\arrowfill@{}\Relbar\Rightarrow}{#1}{#2}}
\makeatother
\newcommand{\cdt}{\!\cdot\!}				% less space around \cdot
\def \dens {\tfrac{\kilo\gram\,}{\metre^3}}		% unit of density: kg/m³

\newcommand{\with} {\qquad\text{\dfe{mit}{avec}{with}}\qquad}
\newcommand{\median}[1]{\ensuremath{\langle {#1} \rangle}}
\newcommand{\bra}[1]{\ensuremath{\langle {#1}|}}
\newcommand{\ket}[1]{\ensuremath{|{#1}\rangle}}
\newcommand{\braket}[2]{\ensuremath{\langle {#1}|{#2}\rangle}}

\def \GP {Grundpraktikum der Physik}
\def \FP {Fortgeschrittenenpraktikum}
\def \TP {TP de physique }				% ajouter la matière
\def \Ces {TP Césire }

% titlepage and table of contents

\newif\iflot
\newcommand{\Titelseite}[8][\lottrue] {
\dfe{\selectlanguage{ngerman}}{\selectlanguage{frenchb}}{\selectlanguage{british}}
\begin{titlepage}
	{\flushright
	\includegraphics[width=\textwidth]{../logo_50.jpg} \\}
	%logo of your university, width might have to be adjusted
	\vspace{5em}
	{\centering
	{\huge \textbf{#2} } \\
	\vspace{4em}
	{\large \textbf{\textsf{\dfe{Versuch}{expérience}{experience} #3:~#4}} } \\
	\vspace{2em}
	{\large \dfe{Durchführung}{exécution}{execution}: #5 \\
	#6: \today \\} }
	
	\vspace{3em}
	{\large
	\begin{tabbing}
		\Large \dfe{Gruppe}{groupe}{group} \pgroup: \\[1mm]
		\studenti \qquad~~ \= \texttt{\emaili}\\
		\studentii  \> \texttt{\emailii}\\
		\ifdefined\studentiii		\studentiii  \> \texttt{\emailiii}\\ \fi
		\\
		\Large \dfe{Betreuer}{enseignant}{supervisor}:\\[1mm]
		#7 \qquad \> \texttt{#8}
	\end{tabbing}}
	\vspace{2em}

	\dfe
	{Wir bestätigen hiermit, dieses Protokoll selbstständig erarbeitet zu haben und um dessen gesamten Inhalt zu wissen. Zur Ausarbeitung wurden ausschließlich die angegebenen Quellen und Hilfen in Anspruch genommen.}
	{Nous attestons d'avoir produit ce compte rendu nous-mêmes. Nous n'avons utilisé que les ressources et sources indiquées.}
	{We hereby attest having written this lab report ourselves. We used no sources aside from those we indicated.}

	\vspace{2em}
	$\overline{\makebox[4.8cm][c]{\raisebox{0pt}[3ex]{\studenti}}}
	\ifdefined\studentiii \hspace{1cm} \else \hfill \fi
	\overline{\makebox[4.8cm][c]{\raisebox{0pt}[3ex]{\studentii}}}
	\ifdefined\studentiii \hspace{1cm}
	\overline{\makebox[4.8cm][c]{\raisebox{0pt}[3ex]{\studentiii}}} \fi$

	\thispagestyle{empty}
	\setcounter{page}{0}
\end{titlepage}


\tableofcontents
\listoffigures
#1 \iflot \listoftables \fi	% lot can be suppressed by empty optional parameter
\enlargethispage{5\baselineskip}
\newpage
}

\graphicspath{{./graphics/}}

\usepackage{latexsym}
\usepackage{amsfonts}
\usepackage{float}
\usepackage[counterclockwise]{rotating}


\begin{document}
\Titelseite{\FP}{26}{Hall-Effekt in Halbleitern}{23. \& 30. Oktober 2014}{Erstabgabe: 12. November 2014\\ Zweitabgabe}{Wolfgang Limmer}{wolfgang.limmer@uni-ulm.de}

% !TeX encoding = UTF-8
% !TeX root = V26_Halleffekt.tex
% !TeX spellcheck = de_DE_frami

\section{Einleitung}
Die Rasterkraftmikroskopie (engl. Atomic Force Microscopy, AFM) ist eine Methode, das Profil einer Probe in einem kleinen Bereich zu vermessen, der je nach Gerät von einem $\milli\metre^2$ bis hinab zu einzelnen Molekülen und Atomen reicht. Hierbei tastet eine Spitze an einem Cantilever die Probe ab und ermittelt die Kraft, die diese auf die Spitze ausübt.

Im folgenden werden einige der wichtigsten Messmodi erklärt, die neben dem Profil auch weitere Eigenschaften des Materials an der Oberfläche ergeben; so z.B. den E-Modul und die Adhäsion.

Weiterhin erstellen wir Aufnahmen verschiedener Proben, um Aussagen über deren Struktur sowie Probleme der AFM treffen zu können. 

\newpage
\section{Theorie}
\subsection{Wechselwirkungspotentiale}
Die zeitlichen Fluktuationen in der Elektronenwolke eines Atoms sind zufällig, wodurch der negative Ladungsschwerpunkt des Atoms mit dem positiv geladenen Atomkern zusammenfällt. Folglich kann das fluktuierende elektrische Feld an sich keine Kraft auf andere Ladungen ausüben. Betrachtet man allerdings mehrere Atome, so induziert das kurzlebige Dipolmoment $\vec p$ in den anderen Atomen ein gleichgerichtetes Dipolmoment $\vec p_i \sim \vec p$. Es bildet sich zu jedem Zeitpunkt eine  vorherrschende Orientierung der $\vec p_i$ heraus, was eine anziehende Kraft bedingt, die Van-der-Waals-Kraft, welche mit dem Atomabstand$r$ gemäß $r^{-7}$ abfällt. Daraus ergibt sich ein Potential der Form $r^{-6}$.

Bei zu starker Annäherung der Atome überwiegen abstoßende Kräfte, die aus der Coulomb-Kraft $F_C$ und der Pauli-Abstoßung zwischen den sich durchdringenden Elektronenwolken folgen. Das zugehörige Potential fallen in etwa mit $r^{-12}$ ab.

Die Wechselwirkungs lässt sich folglich typischerweise relativ gut durch das Lennard-Jones 6-12-Potential beschreiben, hierbei ist $c$ eine Konstante und $r_0$ der Gleichgewichtsabstand der Atome:
\begin{equation}
  V_{LJ} = c \cdot \left[ \left(\tfrac{r_0}{r}\right)^{12} - 2 \cdt \left(\tfrac{r_0}{r}\right)^{6} \right]
\end{equation}

Im Falle der AFM betrachten wir die Wechselwirkung zwischen einer näherungsweise ebenen Probe und einer als Halbkugel\footnote{Für hochauflösende Messungen kann es sich um eine einzelnes Atom an der Spitze handeln.} angenommenen Spitze. Für diese Anordnung ergibt sich für das Van-der-Waals-Potential eine Abhängigkeit von $1/h$ \cite[S. 5]{lit:hampp}, wobei $h$ die Höhe der Spitze über der Probe bezeichnet. Insgesamt erhalten wir folglich mit zwei Konstanten $A, B$ ein Potential
\begin{equation}
  V =  - A \cdt h^{-1} + B \cdt h^{-12} \label{eq:V_Spitze}
\end{equation}

\subsection{Mess-Modi}
Unterhalb des Gleichgewichtsabstands $z_0$ wirken abstoßende Kräfte. Man bezeichnet dies als contact mode, da die Atome anschaulich betrachtet aneinander stoßen. Für $z > z_0$, den sogenannten non contact mode, wird die Spitze zur Probe gezogen. Eine dritte Anordnung ist der intermittent contact mode, bei dem die Spitze einer hochfrequenten Schwingung unterliegt und periodisch die Probe berührt. Dabei lässt sich aus der Modulation der Schwingungsfrequenz die Dämpfung durch die Probe und somit deren lokales E-Modul ermitteln.

Weiterhin wird im contact mode üblicherweise eine der folgenden zwei Konfigurationen verwendet (Bei Proben mit extremer Topologie wird bei constant height das Abbrechen der Spitze riskiert):
\begin{description}
 \item[constant height]: Die Position $z$ des Cantilevers bleibt konstant, das Messsignal entspricht der gemessenen Kraft.
 \item[constant force]: Die Position $z$ des Cantilevers wird nachgesteuert, um die Kraft auf einem vorgegebenen Wert zu halten.
\end{description}

In Abbildung \ref{fig:Potential} sind das auf die Spitze wirkenden Potential aus Gleichung \eqref{eq:V_Spitze} sowie die für verschiedene Messmodi genutzten Bereiche des Potentials dargestellt.

\begin{figure}[h]
	\centering
	\includegraphics[width=0.5\textwidth]{AFM-fig3.png}
	\caption{Potential der Probe für eine Spitze in Funktion des Abstands.}
	\label{fig:Potential}
\end{figure}

%%%%%%%%%%%%%%%%%%%%%%%%%%%%%%%%%%%%%%%%%%%%%%%%%%%%
% Abbildung mit ``Force'' betitelt, im Text als Potential verwendet - korrigieren!
%%%%%%%%%%%%%%%%%%%%%%%%%%%%%%%%%%%%%%%%%%%%%%%%%%%%





\subsection{Kraft-Distanz-Kurve}
Wie unterscheiden sich allerdings die Ergebnisse von Messungen mit verschiedenen Modi? Dies wird am besten klar, wenn man die schematische Kraft-Distanz-Kurve in Abbildung \ref{fig:Fz} betrachtet. 

\begin{figure}[h]
	\centering
	\includegraphics[width=0.5\textwidth]{}
	\caption{Kraft-Distanz-Kurve}
	\label{fig:Fz}
\end{figure}







\begin{figure}[h]
	\centering
	\includegraphics[width=0.5\textwidth]{regelkreis.png}
	\caption{Schematischer Aufbau des Regelkreises eines Rasterkraftmikroskops.}
	\label{fig:feedback}
\end{figure}

\subsection{}

% !TeX encoding = UTF-8
% !TeX root = V34_TEM.tex
% !TeX spellcheck = de_DE_frami

Unterfokus: Objekt unter Fokusebene (durch geringere Brennweite = stärkere Anregung der Spulen), im Bild 1. Saum dominierend hell (Fresnelsaum)

Überfokus analog

Diffractogramm: Fouriertrafo der Abbildung (aber wegen überlagerung mit ... schlechteres bild als im echten beugungsmodus)

Dunkler Ring im Diffractogramm normal, bei zweizähligem Astigmatismus wäre es zur Ellipse verzerrt, wird durch 2 Stigmatoren (Multipole) korrigiert


Aufnahmen:
Unter-, Über-, Fokus (F, Ü, U)
High resolution (mehrere bilder mit Netz ebenen)
Beugung (verdeckter Null Strahl zum Schutz der Kamera)
Bright, Dark Field

Diffractogramme nicht auf USB Stick dabei: freies Programm ImageJ, wahrscheinlich auch matlab (2D fft)

Meist nur eine Schar Netzebenen sichtbar (anstatt kariert) da die meisten Körner nicht in der fokusebene orientiert sind und somit nur eine Achse passt


Beugung: gewisse ringe auf unterschiedlichen Bildern an besten sichtbar, überlagern (entweder nur Messwerte oder Bilder)
Bei 340mm Kamera abstand (30000x Vergrößerung)
300mm Schirm Abstand (26500x Vergrößerung)

Brightfield: objektivblende lässt nur Hauptstrahl durch
Darkfield: ... nur Ausschnitt aus erstem beugungsRing durch
Zwei Bilder für unterschiedlichen Azimutwinkel --> unterschiedliche cluster leuchten auf (zwei verschiedene Orientierungen)




2. Probe
- convergent beam (Kreise mit gebogenem Linien, geben Aufschluss über dicke der probe)
- Beugung, gleiche Einstellungen wie bei der ersten probe

110 Orientierung
1-1+-1, -11+-1, ..., 002, ..., 004 Reflexe zu sehen



Bei der ersten probe
Träger = Kupfernetz + Kohlenstoff-Folie
Probe = kristallines Pulver (durch verdunsten des lösungsmittels gleichmäßig verteilt)

Erste probe Edelmetall (zu bestimmen, vermutlich Gold), zweite probe Si Kristall





\begin{thebibliography}{10}
\bibitem{lit:manual} W. Limmer: Anleitung zum Versuch \emph{Hall-Effekt in Halbleitern}. Universität Ulm, Ulm 2014.\\
\url{http://www.uni-ulm.de/fileadmin/website_uni_ulm/nawi.inst.235/Lehre/FP/gesch%C3%BCtzte_Dateien/Anleitung_Halleffekt.pdf}
\bibitem{lit:keithley} Keithley Instruments Inc.: \emph{Model 7065 Hall Effect Card}.\\
\url{www.keithley.com/data?asset=560}
\bibitem{lit:Iba09} H. Ibach und H. Lüth: \emph{Festkörperphysik}. 7. Auflage, Springer Verlag, Berlin (2009)
\bibitem{lit:Sze07} S. M. Sze und Kwok K. Ng: \emph{Physics of semiconductor devices}. 3. Auflage, John Wiley \& Sons, Inc., New Jersey (2007)
\bibitem{lit:GroMa14} R. Gross und A. Marx: \emph{Festkörperphysik}. 2. Auflage, Walter de Gruyther, Berlin (2014)
\bibitem{lit:Wiki1} Wikipedia-Benutzer Locusta: \emph{Geometriefaktor f für Van-der-Pauw-Methode}. Wikimedia Commons, CC BY-SA 3.0 2010\\ \url{http://commons.wikimedia.org/wiki/File:Pauw_Korrektur.svg}
\end{thebibliography}

\newpage
\section{Anhang}
\vspace{2em}
\begin{addmargin}[-3em]{3em}
\lstinputlisting[style=mlab, caption=%
\url{https://github.com/aquileia/FP-WS2014/blob/master/V26 Hall-Effekt/Matlab/evaluation.m}\label{code}\\
]{./Matlab/evaluation.m}
\end{addmargin}

\end{document}