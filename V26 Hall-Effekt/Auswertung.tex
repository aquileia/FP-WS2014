% !TeX encoding = UTF-8
% !TeX root = V26_Halleffekt.tex
% !TeX spellcheck = de_DE_frami

\section{Durchführung}
Wir untersuchen im Folgenden ein $d = 0,31\,\micro\metre$ dickes Scheibchen aus Bor-dotiertem Silizium. Da \textsf{B} 3 Valenzelektronen besitzt, \textsf{Si} aber 4, können diese Löcher im sonst vierbindigen Kristallgitter in das Valenzband übergehen -- der Halbleiter ist folglich p-dotiert.

Anstatt die Kontaktstellen klein zu halten, vermessen wir eine Probe in Kleeblatt-Form; dabei handelt es sich um einen Wafer mit vier radialen Einschnitten. Durch die höhere Stromdichte im Zentrum dominiert dieses die elektrischen Eigenschaften der Probe, wodurch auf den vier \glqq Blättern\grqq{} relativ große Kontakte platziert werden können, ohne die Messung zu stören. Zudem sind die Abstände zweier Kontakte relativ scharf definiert, da der Strom immer über die Mitte fließt.

Bevor die Probe vermessen werden kann, muss allerdings das Magnetfeld und die Ausrichtung der Probe zu diesem kalibriert werden; die Berechnung geht von einem senkrecht zur Probe angelegten Feld aus.

\subsection{Vorbereitung}
% ...


\subsection{Messung von $\rho$, $\mu$ und $p$}

Die Apparatur der Firma Keithley nutzt eine Schalt-Matrix aus Relais, wodurch alle vier benachbarten Kontakt-Paare in beide Stromrichtungen beschaltet werden können. Somit ergeben sich 8 Mess-Konfigurationen für die Bestimmung des spezifischen Widerstands $\rho$.\\
Die Nummerierung der gemessenen Spannungen $V^R_1, \dots, V^R_8$ ist in Tabelle 3-3 der Versuchanleitung ersichtlich, wird hier aber nicht nochmals wiedergegeben, da die Messung ohnehin automatisiert erfolgt.

Unter Kenntnis der Dicke $d$ des Plättchens, des Formfaktors $f \approx 1$ und des Stroms  $I = 100\,\micro\ampere$ erhält man
\begin{equation}
\rho ~=~ 1,1331 \,\centi\metre^{-4} \cdot \frac{f \cdt d}{2 \cdt I} ~ \sum\limits_{i=1}^8 (-1)^i \; V^R_i
\end{equation}

Da es nur zwei \glqq diagonale\grqq{} (gegenüberliegende) Verbindungen gibt, erhält man inklusive der Umpolung des Stroms 4 Beschaltungen für den Halleffekt. Während vorgesehen ist, die Messung nach Umpolen des Magnetfeldes zu wiederholen, genügen bereits diese vier Messungen, um die Offset-Spannung aus Gleichung \eqref{eq:U_offset} zu eliminieren. Diese Vereinfachung reduziert den Aufwand für die Messung beträchtlich. Die Zuordnung von $V^H_1, \dots, V^H_4$ erfolgt gemäß Tabelle 3-4 der Anleitung.

Zieht man neben $d$ und $I$ (gleichbleibend zur vorherigen Messung) noch das zuvor eingestellte Magnetfeld $B$ hinzu, so ergibt sich für den Hallkoeffizienten 
\begin{equation}
R_H ~=~ \frac{0,25 \cdt d}{B \cdt I} ~ \sum\limits_{i=1}^4 (-1)^i \; V^H_i
\end{equation}
Da $R_H$ typischerweise in $\tfrac{\centi\metre^3}{\ampere\second}$ und $d$ in cm angegeben wird, ergibt sich in diesem Fall ein Vorfaktor von 2500 anstelle 0,25.

Die Beweglichkeit ergibt sich zu
\begin{equation}
\mu ~=~ \frac{|R_H|}{\rho}
\end{equation}
und mit Gleichung \eqref{eq:p_rho_mu} erhalten wir schließlich die Löcherkonzentration
\begin{equation}
p = (e \cdt \mu \cdt \rho)^{-1}.
\end{equation}

\newpage
\section{Auswertung}

\subsection{Induktivität des Elektromagneten}
%...
% B = 0.038 kg/C^2  \cdot  I  +  0.012 T

Da wir die angestrebte Magnetfeldstärke von $B = 0,5 \,\tesla$ nicht genau erreichen müssen, wählen wir den nächstliegenden Datenpunkt: Bei 13\,A erzeugt der Elektromagnet ein Feld von 0,509\,T.

\subsection{Kennlinie der Probe}
% U = 4.4388 \ohm  \cdt  I  +  0.8125 V
\subsection{Ausrichtung im Magnetfeld}

\subsection{Temperatur-Abhängigkeit elektrischer Eigenschaften} 
Mithilfe eines eigens erstellten Matlab-Skripts (Listing \ref{code}) werten wir die erfasste Datenreihe von 80 bis 300\,K aus.

\begin{figure}[p]
\centering
% This file was created by matlab2tikz v0.5.0 (commit ea8b8d55b483b22df88d8f84812a33c02f2c983f) running on MATLAB 8.2.
%Copyright (c) 2008--2014, Nico Schlömer <nico.schloemer@gmail.com>
%All rights reserved.
%Minimal pgfplots version: 1.3
%
\begin{tikzpicture}

\begin{axis}[%
width=0.95\textheight,
height=0.9\textwidth,
at={(0\textwidth,0\textwidth)},
scale only axis,
xmin=0,
xmax=12.5,
xtick={0,1,...,12},
xlabel={inverse Temperatur 1000/$T~(\kelvin^{-1})$},
ymode=log,
ymin=9e+16,
ymax=5e+18,
yminorticks=true,
ylabel={Löcherdichte $p~(\centi\metre^{-3})$},
legend style={legend cell align=left}
]
\addplot [color=blue,line width=1.5pt,only marks,mark=o,mark options={solid}]
  table[row sep=crcr]{%
12.1951219512195	9.19799779848282e+16\\
11.7647058823529	1.17955867390416e+17\\
11.1111111111111	1.4274653416398e+17\\
10.752688172043	1.5495041137186e+17\\
10.5263157894737	1.66011760889568e+17\\
10.2040816326531	1.78824158579839e+17\\
9.70873786407767	2.04755334491956e+17\\
9.52380952380952	2.13769098768795e+17\\
9.09090909090909	2.37201786429122e+17\\
8.62068965517241	2.64418810091079e+17\\
8.33333333333333	2.84584797648644e+17\\
8	3.08925719934262e+17\\
7.69230769230769	3.3266174738184e+17\\
7.40740740740741	3.55455166913931e+17\\
7.14285714285714	3.77703078763305e+17\\
6.89655172413793	4.01244015720498e+17\\
6.66666666666667	4.23186486018773e+17\\
6.45161290322581	4.46153455380262e+17\\
6.25	4.66044871765005e+17\\
6.06060606060606	4.871604603705e+17\\
5.88235294117647	5.08040060433009e+17\\
5.71428571428571	5.2513537076036e+17\\
5.55555555555556	5.4485955195137e+17\\
5.40540540540541	5.64034287112979e+17\\
5.23560209424084	5.85451678118927e+17\\
5.12820512820513	6.01863076074571e+17\\
5	6.18359582285918e+17\\
4.8780487804878	6.3591565369122e+17\\
4.76190476190476	6.50819132147362e+17\\
4.65116279069767	6.67608785127859e+17\\
4.54545454545455	6.84257466685859e+17\\
4.44444444444444	6.98806536228652e+17\\
4.34782608695652	7.09693062269456e+17\\
4.25531914893617	7.23382946926613e+17\\
4.13223140495868	7.43746887791472e+17\\
4.08163265306122	7.52067013626841e+17\\
4	7.67461404792935e+17\\
3.92156862745098	7.79845320564008e+17\\
3.83141762452107	7.96514277617866e+17\\
3.77358490566038	8.07313619865247e+17\\
3.7037037037037	8.22848236983129e+17\\
3.63636363636364	8.36132714081887e+17\\
3.57142857142857	8.46884053634095e+17\\
3.50877192982456	8.61342745662246e+17\\
3.44827586206897	8.7422796257879e+17\\
3.43642611683849	8.77414200451798e+17\\
3.38983050847458	8.88389875446015e+17\\
3.36700336700337	8.81662859581089e+17\\
3.33333333333333	9.04913130402291e+17\\
};
\addlegendentry{Messwerte};

\addplot [color=green,dotted,line width=2.0pt]
  table[row sep=crcr]{%
1	1.55078081815462e+17\\
0.833333333333333	6.198706564364e+17\\
0.714285714285714	1.72864618284833e+18\\
0.625	3.83206324996288e+18\\
0.555555555555556	7.26779408678282e+18\\
0.5	1.23319506928369e+19\\
};
\addlegendentry{Gleichung \eqref{eq:p_intrinsic}};

\addplot [color=black,solid,line width=2.0pt]
  table[row sep=crcr]{%
12.5892541179417	9.90935758037624e+16\\
11.6774210239705	1.22283341228078e+17\\
10.83163152428	1.48980778625418e+17\\
10.0471021158647	1.79331736575803e+17\\
9.31939576234078	2.13425212288967e+17\\
8.64439679955028	2.51287431758695e+17\\
8.0182876587383	2.92877727026951e+17\\
7.43752727565905	3.38086365361204e+17\\
6.89883106349873	3.86734411986532e+17\\
6.39915233634927	4.38575640272211e+17\\
5.93566507817007	4.93300442580418e+17\\
5.50574795978489	5.50541641165461e+17\\
5.10696951351927	6.09882049866057e+17\\
4.73707438163119	6.7086359120366e+17\\
4.39397056076079	7.32997727069207e+17\\
4.07571757025783	7.95776912416577e+17\\
3.78051547747108	8.58686730002232e+17\\
3.50669471793018	9.21218312492164e+17\\
3.25270665284652	9.82880611447386e+17\\
3.0171148105293	1.04321203880153e+18\\
2.79858676218128	1.10179099518421e+18\\
2.5958865861264	1.15824482039629e+18\\
2.40786787784943	1.21225676146198e+18\\
2.23346726631486	1.26357065478168e+18\\
2.07169839989531	1.31199315585329e+18\\
1.9216463678961	1.35739351038964e+18\\
1.78246252612584	1.3997010262605e+18\\
1.65335969724827	1.43890055588753e+18\\
1.53360771877001	1.47502641462686e+18\\
1.42252931348537	1.50815522908786e+18\\
1.31949625902256	1.53839822527335e+18\\
1.22392583482776	1.56589343351775e+18\\
1.13527752649218	1.59079821606145e+18\\
1.05304996878311	1.6132824283326e+18\\
0.976778110089488	1.63352242186158e+18\\
0.906030582245338	1.65169599835156e+18\\
0.840407260855403	1.66797834009252e+18\\
0.779537002325197	1.68253887667639e+18\\
0.723075544796735	1.69553900307412e+18\\
0.670703561118431	1.70713053798139e+18\\
0.622124852837335	1.71745480067486e+18\\
0.577064674999591	1.72664218554231e+18\\
0.535268182284711	1.7348121221455e+18\\
0.496498987685534	1.74207332186972e+18\\
0.460537825582242	1.74852422736164e+18\\
0.427181311649226	1.75425359624996e+18\\
0.396240792581251	1.75934116492908e+18\\
0.367541279133348	1.76385835082311e+18\\
0.340920456440076	1.7678689622752e+18\\
0.316227766016838	1.77142989400972e+18\\
};
\addlegendentry{Gleichung \eqref{eq:density_p}};

\addplot [color=red,dotted,line width=2.0pt]
  table[row sep=crcr]{%
12.5892541179417	1.01939275366965e+17\\
11.6774210239705	1.26661387413179e+17\\
10.83163152428	1.55557311123113e+17\\
10.0471021158647	1.88993271484791e+17\\
9.31939576234078	2.27327355761909e+17\\
8.64439679955028	2.70907862020172e+17\\
8.0182876587383	3.20072494020932e+17\\
7.43752727565905	3.7514835874072e+17\\
6.89883106349873	4.36452693585815e+17\\
6.39915233634927	5.04294231126609e+17\\
5.93566507817007	5.78975098987094e+17\\
5.50574795978489	6.607931500195e+17\\
5.10696951351927	7.50044621475216e+17\\
4.73707438163119	8.47027029925705e+17\\
4.39397056076079	9.52042219678892e+17\\
4.07571757025783	1.06539949506583e+18\\
3.78051547747108	1.1874187801696e+18\\
3.50669471793018	1.31843376251493e+18\\
3.25270665284652	1.45879498935016e+18\\
3.0171148105293	1.6088728960573e+18\\
2.79858676218128	1.76906075571698e+18\\
2.5958865861264	1.93977754686214e+18\\
2.40786787784943	2.12147074300732e+18\\
2.23346726631486	2.31461903273411e+18\\
2.07169839989531	2.51973498308948e+18\\
1.9216463678961	2.73736766197054e+18\\
1.78246252612584	2.96810523719729e+18\\
1.65335969724827	3.21257757127566e+18\\
1.53360771877001	3.47145883158014e+18\\
1.42252931348537	3.74547013597504e+18\\
1.31949625902256	4.03538225386446e+18\\
1.22392583482776	4.34201838241517e+18\\
1.13527752649218	4.66625701731672e+18\\
1.05304996878311	5.00903493699765e+18\\
0.976778110089488	5.37135031875832e+18\\
0.906030582245338	5.7542660048514e+18\\
0.840407260855403	6.15891293617114e+18\\
0.779537002325197	6.58649377092404e+18\\
0.723075544796735	7.0382867054626e+18\\
0.670703561118431	7.5156495143796e+18\\
0.622124852837335	8.0200238269898e+18\\
0.577064674999591	8.5529396574706e+18\\
0.535268182284711	9.11602020619467e+18\\
0.496498987685534	9.71098695016499e+18\\
0.460537825582242	1.03396650409533e+19\\
0.427181311649226	1.1003989029146e+19\\
0.396240792581251	1.17060089350141e+19\\
0.367541279133348	1.2447896685942e+19\\
0.340920456440076	1.32319529420763e+19\\
0.316227766016838	1.40606143326828e+19\\
};
\addlegendentry{Gleichung \eqref{eq:p_approx}};

\end{axis}
\end{tikzpicture}%
\caption{Löcherdichte $p(T)$}
\label{fig:plot_p}
\end{figure}

Die Fit-Parameter zur Modellierung der Messdaten mit Gleichung \eqref{eq:density_p} ergeben $N_A$ und $E_A$, also die Konzentration und das Energieniveau der Akzeptoren. 

Wie aus Abbildung \ref{fig:plot_p} ersichtlich, bleibt das Plateau der Störstellensättigung sowohl in den Messdaten als auch in dem Plot von Gleichung \eqref{eq:density_p} nahezu aus. Dies ist durch die hohe Konzentration $N_A = 1,8\cdt 10^{18}\,\centi\metre^{-3}$ an Dotierungsatomen bedingt, aufgrund der auch bei Raumtemperatur noch einige nicht-ionisierte Bor-Atome verbleiben.

Zudem nähert sich bei Erhöhung von $N_A$ auch das Akzeptorniveau dem Valenzband an (Dies ist auch der Grund, warum in Abbildung \ref{fig:fermi_density} die Fermienergie mit $N_A$ sinkt). Dabei sinkt der Abstand $E_A$ der Niveaus von den für Bor typischen 45\,meV auf einen Wert von nunmehr 31,7\,meV.

\begin{figure}[p]
\centering
% This file was created by matlab2tikz v0.5.0 (commit ea8b8d55b483b22df88d8f84812a33c02f2c983f) running on MATLAB 8.2.
%Copyright (c) 2008--2014, Nico Schlömer <nico.schloemer@gmail.com>
%All rights reserved.
%Minimal pgfplots version: 1.3
%
\begin{tikzpicture}

\begin{axis}[%
width=0.9\textheight,
height=0.9\textwidth,
at={(0\textwidth,0\textwidth)},
scale only axis,
xmode=log,
xmin=45,
xmax=300,
xtick={63,100,158,251},
xlabel={Temperatur $T$ (K)},
ymode=log,
ymin=120.5,
ymax=285,
ytick={126,158,200,251},
ylabel={Beweglichkeit $\mu~(\centi\metre^2/\volt\second)$},
legend style={legend cell align=left}
]
\addplot [color=blue,line width=1.5pt,only marks,mark=o,mark options={solid}]
  table[row sep=crcr]{%
82	244.300300786938\\
85	252.786090721705\\
90	257.314267765106\\
93	258.129158219425\\
95	258.83111566487\\
98	259.585845209269\\
103	259.640846407217\\
105	259.657095837514\\
110	258.402830105037\\
116	256.455524871269\\
120	254.899951737343\\
125	252.229637785867\\
130	249.438390014947\\
135	245.912203805534\\
140	242.657820692944\\
145	238.634802456646\\
150	234.601932529588\\
155	230.337459728688\\
160	226.423702869282\\
165	222.082210044402\\
170	217.545064758772\\
175	213.896451558582\\
180	209.410247748308\\
185	204.951506445784\\
191	199.838051167483\\
195	195.988318004159\\
200	191.898826712964\\
205	187.584927845995\\
210	183.862252089627\\
215	179.589511412717\\
220	175.326027897898\\
225	171.571284686148\\
230	168.669589382853\\
235	165.078963472015\\
242	159.786990298345\\
245	157.586999917893\\
250	153.62596980649\\
255	150.425062314759\\
261	146.201300910057\\
265	143.415598869528\\
270	139.590559929408\\
275	136.377060066437\\
280	133.718235317397\\
285	130.408961483718\\
290	127.429432820283\\
291	126.741540561473\\
295	124.408823547897\\
297	125.720492954517\\
300	121.111846018363\\
};
\addlegendentry{Messwerte};

\addplot [color=black,solid,line width=1.5pt]
  table[row sep=crcr]{%
170	284.22820174932\\
300	121.243556529821\\
};
\addlegendentry{$c_1 \cdot T^{-3/2}$};

\addplot [color=red,solid,line width=1.5pt]
  table[row sep=crcr]{%
46.5	120.49333529287\\
83	287.343075086211\\
};
\addlegendentry{$c_2 \cdot T^{3/2}$};

\end{axis}
\end{tikzpicture}%
\caption{Beweglichkeit $\mu(T)$}
\label{fig:plot_mu}
\end{figure}

\newpage
\begin{addmargin}[-3em]{3em}
\lstinputlisting[style=mlab, caption=%
\url{https://github.com/aquileia/FP-WS2014/blob/master/V26 Hall-Effekt/Matlab/evaluation.m}\label{code}
]{./Matlab/evaluation.m}

\end{addmargin}
